\documentclass[a4paper,11pt]{article}

\usepackage[english]{babel}

\usepackage[left=2.5cm, right=2.5cm]{geometry}

\setlength{\parindent}{0em}

\begin{document}


 The material contained in this Documentation
is in part also contained in the function \texttt{example.ci} of the
package.

\section{Main and interaction effect analysis}


Organise your data in a \texttt{data.frame} X. Eg.:

\begin{verbatim}
instance algorithm class results
inst1    alg1      boh1   40
inst1    alg2      boh1   40
...
inst1    alg1      boh2   40
inst1    alg2      boh2   40
...
\end{verbatim}

\texttt{instance, algorithm} and \texttt{class} must be factors. Note
that the columns names can be redefined but \texttt{class} must remain.

\begin{verbatim}
> formula <- results ~ instance * algorithm
\end{verbatim}

See R documentation for formula syntax (above all on the difference
between the sign ``*'' and ``+'' in the formula).

\bigskip
Parametric analysis:
\begin{verbatim}
> aov(formula,data=X)
\end{verbatim}

No need for the library so far. Everything is already in R.

\medskip

Then load the library:

\begin{verbatim}
> source("lib_effects.R")
\end{verbatim}

 
Nonparametric analysis (by ranks):
\begin{verbatim}
>  Friedman.test(formula,data=X)
\end{verbatim}

Nonparametric permutation analysis :
\begin{verbatim}
> permut.aov(formula,data=X,method="same",B=IN.PERM,type="csp")    
\end{verbatim}

Values \texttt{eff1} \texttt{eff2} \texttt{inter} represent main inteeractions and
intereaction effects, respectively.

\section{Post-hoc analysis}

The packages \texttt{lattice} e \texttt{Hmisc} must be installed.

Organise your data in a \texttt{data.frame} X. Eg.:

\begin{verbatim}
instance algorithm class results
inst1    alg1      boh1   40
inst1    alg2      boh1   40
...
inst1    alg1      boh2   40
inst1    alg2      boh2   40
...
\end{verbatim}

instance, algorithm and class must be factors.

Then from R, load the library:

\begin{verbatim}
> source("lib_posthoc.R")
\end{verbatim}

for a parametric analysis:

\begin{verbatim}
> R <-AP.SCI(results~algorithm*instance,data=X,method="Parametric",test="Tukey")
> plot.ci(R)
\end{verbatim}

Note that, in the formula above \texttt{algorithm * instance} accounts
also for a significant interaction while \texttt{algorithm + instance}
does not account for interaction.

\smallskip

In the nonparametric case:

\begin{verbatim}
> R<-AP.SCI(results~algorithm+instance,data=X,method="Ranks",test="Conover")
> plot.ci(R)
\end{verbatim}

Interaction cannot be handled in this case. The output will be in a file
named \texttt{plot.ci.eps}. For
improving graphical layout edit the function \texttt{plot.ci.xclass}.



\bigskip

If instead the analysis includes more than one class of instances that
for some reason one wants to maintain distinct:

\begin{verbatim}
> R <- AP.SCI.wrapper.class(results~algorithm*instance,data=X,
                            method="Parametric",test="Tukey")
> plot.ci.xclass(R)
\end{verbatim}

or

\begin{verbatim}
> R <- AP.SCI.wrapper.class(results~algorithm+instance,data=X,
                            method="Ranks",test="Conover")
> plot.ci.xclass(R)
\end{verbatim}

The output will be in a file named \texttt{plot.ci.xclass.eps}. For
improving graphical layout edit the function \texttt{plot.ci.xclass}.



\subsection{Complete nomenclature}

\begin{description}
\item[\texttt{formula}:] ...
\item[\texttt{method}:] \texttt{"Parametric", "Ranks", "Permutations"}
\item[\texttt{test}:] Parametric case: \texttt{"Tukey", "LSD",
    "LSDBonferroni", "Sheffe"}\\ Nonparametric case:
  \texttt{"Conover","Sheskin","Permutations"}, i.e.,  Friedman test from
  Conover or Sheskin or permutation test on ranks.
\item[\texttt{alpha}:] is the family or experiment wise error rate. 
\item[\texttt{adj.method}:] only in Permutation method.
  \texttt{"simul.Bonferroni", ``none'' "Bonferroni"}
\item[\texttt{type}:] \texttt{"csp","usp"} Constrained and Unconstrained
  Synchrnised Permutations
\item[\texttt{B}:] only in Permutation method. Numer of Monte Carlo
  Samples from the permutation space.
\end{description}

\medskip
Values 
\begin{description}
\item \texttt{lower} and \texttt{upper}
\end{description}

 %   plot.hsu(Y,file=paste(ex,"-csp-",wh,sep=""),"CSP",dev="eps")
    

\end{document}